\documentclass{pbmlbook}
\usepackage{lingex}

\begin{document}

\chapter*{Příkladové věty jedním příkazem}

Základní příkaz je

\noindent\verb \exsent[<typ>]{veta} ,

který uvede číslovanou příkladovou větu.
V závislosti na parametru \textit{<typ>} je označena
číslem (parametr ``\texttt{0}''), nebo číslem a písmenem (``\texttt{a}'').
Číslovaný příklad má číslo o jedna vyšší než předchozí;
písmeno také vždy následuje za předchozím použi\-tým písmenem a
je uvedeno spolu s posledním použitým číslem.
Nové číslo resetuje písmena zpět k \textit{a}.
Jednotlivá čísla jdou tedy za sebou například takto:
1, 2, 3, 4, 5, 5a, 5b, 5c, 5d, 6, 6a, 6b, 7, 8, 9, 9a.
Pokud není parametr uvedený, použije se číslovaná varianta.

\begin{verbatim}
\exsent[0]{Chytal tlouště.}
\exsent[a]{Chytal na višni.}
\exsent[a]{Chytal tlouště na višni.}
\exsent{Jak se do lesa volá\ldots}
\exsent[a]{Jak se do lesa volá, tak ti dá hajnej facku.}
\end{verbatim}

\exsent[0]{Chytal tlouště.}
\exsent[a]{Chytal na višni.}
\exsent[a]{Chytal tlouště na višni.}
\exsent{Jak se do lesa volá\ldots}
\exsent[a]{Jak se do lesa volá, tak ti dá hajnej facku.}

Větě lze přiřadit label pro pozdější užití:

\begin{verbatim}
\exsent[0]{\label{oko-chran}Chránit jako oko v hlavě.}
\exsent[a]{\label{oko-opatr}Opatrovat jako oko v hlavě.}
\exsent[0]{\label{preruseni}Rušivá věta.}
\end{verbatim}

\exsent[0]{\label{oko-chran}Chránit jako oko v hlavě.}
\exsent[a]{\label{oko-opatr}Opatrovat jako oko v hlavě.}
\exsent[0]{\label{preruseni}Rušivá věta.}

Label je pak možné použít dvojím způsobem:
odkázat se v textu na zmíněný příklad a
dodatečně parafrázovat danou větu,
ačkoli už jsem mezitím uvedl větu \verb!\ref{preruseni}! \ref{preruseni}:

\begin{verbatim}
Věty \ref{oko-chran} a \ref{oko-opatr} mají ještě méně známé varianty
\refexsent[b]{oko-chran}{Chránit jako zuby v puse.}
\refexsent[c]{oko-chran}{Opatrovat jako zuby v puse.}
\end{verbatim}

Věty \ref{oko-chran} a \ref{oko-opatr} mají ještě méně známé varianty
\refexsent[b]{oko-chran}{Chránit jako zuby v puse.}
\refexsent[c]{oko-chran}{Opatrovat jako zuby v puse.}

K tomu slouží standardní příkaz \verb!\ref{<label>}! a příkaz

\verb!\refexsent[<pismeno>]{<label>}{veta}!.

Jako \textit{<label>} se použije zvolený řetězec a
písmeno udává, jaké písmeno se za číslo připojí.
Pokud se písmeno vynechá, připojí se \textit{a}.
\textbf{Pozor}, písmeno se připojí i k \ref{oko-opatr} a tak by
z \verb!\refexsent[d]{oko-opatr}{CHYBA}! vzniklo:

\refexsent[d]{oko-opatr}{CHYBA}

což typicky není cílem.
V takovém případě patří \verb!\label! k původní příkladové větě
bez písmene: \ref{oko-chran}.

\hrule

\chapter*{Příkladové věty jako prostředí}

Druhá možnost jak použít číslování vět je v prostředí \textit{exe}.
Funkcionalita je úplně stejná, vzhled přibližně stejný, číslování je společné.
Příkazy si odpovídají podle následující tabulky:

\medskip
\begin{tabular}{|l|l|l|}
\hline
           & samostatně                     & v prostředí               \\
\hline
číslovaná  & \verb!\exsent[0]{...}!         & \verb!\ex ...!            \\
abecední   & \verb!\exsent[a]{...}!         & \verb!\exa ...!           \\
odkazovaná & \verb!\refexsent[.]{lab}{...}! & \verb!\exref[.]{lab} ...! \\
\hline
\end{tabular}
\bigskip

Snad postačí příklad:

\begin{verbatim}
\begin{exe}
\ex  Chytal tlouště.
\exa Chytal na višni.
\exa Chytal tlouště na višni.
\ex  Jak se do lesa volá\ldots
\exa Jak se do lesa volá, tak ti dá hajnej facku.
\ex  \label{oko-chran2}Chránit jako oko v hlavě.
\exa \label{oko-opatr2}Opatrovat jako oko v hlavě.
\ex  \label{preruseni2}Rušivá věta. 
\end{exe}

Věty \ref{oko-chran2} a \ref{oko-opatr2}
mají ještě méně známé varianty:
\begin{exe}
\exref[b]{oko-chran2} Chránit jako zuby v puse.
\exref[c]{oko-chran2} Opatrovat jako zuby v puse.
\exref[d]{oko-opatr2} CHYBA
\exref{preruseni2}    Bez písmene se připojuje \textit{a}.
\end{exe}
\end{verbatim}

\begin{exe}
\ex  Chytal tlouště.
\exa Chytal na višni.
\exa Chytal tlouště na višni.
\ex  Jak se do lesa volá\ldots
\exa Jak se do lesa volá, tak ti dá hajnej facku.
\ex  \label{oko-chran2}Chránit jako oko v hlavě.
\exa \label{oko-opatr2}Opatrovat jako oko v hlavě.
\ex  \label{preruseni2}Rušivá věta. 
\end{exe}

Věty \ref{oko-chran2} a \ref{oko-opatr2}
mají ještě méně známé varianty:
\begin{exe}
\exref[b]{oko-chran2} Chránit jako zuby v puse.
\exref[c]{oko-chran2} Opatrovat jako zuby v puse.
\exref[d]{oko-opatr2} CHYBA
\exref{preruseni2}    Bez písmene se připojuje \textit{a}.
\end{exe}


\hrule
\setcounter{lastarab}{0}
\newpage

\section*{Příklady}
\exsent{prvni prikladova veta}
\exsent{\label{druha}druha prikladova veta}
\exsent{\label{treti}treti prikladova veta}
\exsent[a]{varianta treti prikladove vety}
\exsent[a]{\label{tri-be}dalsi varianta treti prikladove vety}
\refexsent[x]{tri-be}{chybne odkazovani na treti vetu}
\refexsent{druha}{\label{druha-ref}spravne odkazovani na druhou vetu}
\exsent[a]{ctvrta varianta treti prikladove vety}
\exsent{ctvrta prikladova veta}
\exsent[0]{pata prikladova veta}
\exsent[a]{varianta pate prikladove vety}
\refexsent[d]{treti}{zpet k treti vete, jeji pata varianta}

Druha veta je \ref{druha}, treti je \ref{treti},
dalsi jeji varianta je \ref{tri-be}.

Veta tvorena odkazem na druhou vetu je \ref{druha-ref}.

Lorem ipsum dolor sit amet, consectetur adipiscing elit. Curabitur eu enim et
magna sodales posuere sed quis felis. Maecenas sit amet placerat augue. Integer
non sapien orci. Cras sapien tellus, laoreet eget blandit id, lobortis id odio.


\begin{exe}
\ex aaa
\ex aaaa
\exa bbb
\exa \label{x:c} ccc
\ex ddd
\exa eee
\end{exe}

a dalsi priklady

\begin{exe}
\ex \label{x:f} fff
\exa ggg
\exa hhh
\ex iii
\exa jjj
\exref{x:c} \label{x:c2} ccc2
\exref[e]{x:f} \label{x:f2} fff2
\end{exe}

Odkaz na ``fff'' je \ref{x:f} a \ref{x:f2} a
odkaz na ``ccc'' je \ref{x:c} a \ref{x:c2}.

Vestibulum sagittis libero eget orci suscipit tempor ut eget erat. Class aptent
taciti sociosqu ad litora torquent per conubia nostra, per inceptos himenaeos.
Fusce lacus nulla, ornare at faucibus sed, pretium sed leo.

Mimochodem, prikladovych vet tvorenych odkazem bylo presne \arabic{refcount}.

\end{document}
